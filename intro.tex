\section{Введение}

Задача трекинга трёхмерного объекта по видеопотоку состоит в определении положения изначально заданного объекта на каждом кадре видео. Отслеживание положения объекта в 3D требуется в приложениях дополненной реальности, во многих задачах робототехники, и многих других приложениях, в которых требуется понимание окружающего простанства по видео. Одно из применений трекинг находит в создании видеоэффектов для кинематографа, где часто требуется отследить на видео некоторый объект с целью придания ему затем дополнительных свойств. (ссылок сюда на примеры использования)

Задача формулируется следующим образом: по данной 3D-модели объекта, видеопотоку и положению объекта относительно камеры на первом кадре получить позиции объекта на всех последующих кадрах. При этом модель объекта считается жёсткой, и его позиция задаётся матрицей поворота-переноса $(R | t)$, которая может быть задана шестью параметрами, три из которых отвечают за поворот объекта, и три -- за его положение относительно начала координат (ссылку сюда). Для того, чтобы получить позицию на каждом следующем кадре, можно извлекать различную иинформацию из изображения. 

Многие алгоритмы трекинга основаны на поиске и отслеживании ключевых точек изображения. Основная их идея состоит в вычислении оптического потока в окрестностях ключевых точек между соседними кадрами (например, алгоритмом Лукаса-Канаде~\cite{LukasKanade}) и вычислении наиболее правдоподобных параметров трансформации (ссылку сюда). Такие методы хорошо работают для отслеживания хорошо текстурированных объектов и могут работать при перемене освещения между кадрами. Примерами таких алгоритмов являются~\cite{LourakisFeatures} (ещё ссылку сюда). 

Тем не менее, на однородных по текстуре объектах может быть очень мало ключевых точек, и в такой ситуации отслеживать положение объекта по ним не получится. Одним из способов преодоления этой проблемы является отслеживание контура объекта. Для этого выделяют границы на изображении и среди них ищут те, что могут относиться к контуру. Примерами таких алгоритмов являются (ссылку сюда), исользующий что-то и (ссылку сюда), идея которого состоит в том, чтобы... Алгоритмы на основе поиска контуров способны отследить слабо текстурированные объекты, но также обладают некоторыми недостатками. По контуру не всегда можно однозначно восстановить позицию объекта, особенно в случае, когда объект симметричен, и трекинг может сильно ошибаться при возникновении таких неоднозначностей.

В последние годы получили активно развиваются алгоритмы, основанные на распределении цветов в отдельных областях изображения (region-based methods). Они собирают статистику того, насколько часто пиксели различных цветов встречаются на проекции объекта и на окружающем фоне. На каждом новом кадре позиция ищется таким образом, чтобы проекция объекта наилучшим образом соответствовала собранной статистике. Одним из первых таких алгоритмов стал PWP3D \cite{PWP3D}, в котором строилась гистограмма цветов на проекции объекта и на фоне. В нём была введена функция ошибки, определяющая апостериорную вероятность позиции в соответствии с гистограммой. Этот алгоритм был развит в работах (ссылки, ссылки), в которых гистограммы строились для отдельных областей, с тем чтобы подсчитывать статистику по цвету локально и повысить устойчивость к зашумлённым и пёстрым изображениям. Эти алгоритмы показывают неплохие результаты (ссылка), решая проблему слабо текстурированных изображений, и частично решая проблему смазывания, но, как и методы на основе контуров, плохо справляются с симметричными объектами, и кроме того неустойчивы к резкой смене освещения. 

Также стоит отметить алгоритмы, собирающие попиксельную информацию, но анализирующих не цвет, а дескрипторы отдельных пикселей (ссылка на Photometric методы).

Таким образом, существуют различные факторы, влияющие на качество трекинга различными алгоритмами: текстурированность объекта, наличие смазывания, симметричность объекта, скорость движения и т. д. Различные алгоритмы по-разному справляются с этими условиями, и там, где один сбивается, другой может работать стабильно. Отдельным фактором, влияющим на выбор алгоритма, является ограничение на время работы, которое может быть различно для разных задач.

В данной работе проведена попытка одновеменно исользовать инфомацию о распределении цветов на проекции и фоне на предыдущих изображениях и информацию о ключевых точках. Подход на основе распределения цветов выбран, так как подходит для любых типов текстур, устойчив при несильном смазывании и показывающий хорошие результаты~\cite{Tjaden2018}. Добавление к нему информации о ключевых точках позволяет разрешить неопределённость, связанную с симметричностью объектов, и кроме того привлекает информацию о внутренней области проекции, тогда как алгоритмы, основанные на цветах, используют в основном информацию о пикселях вокруг контура.

В последнее время предпринимались попытки совместить различные подходы в алгоритмах трекинга. К ним относится, например, ~\cite{Bugaev_2018_ECCV}, где ключевые точки исользуются для инициализации контурного метода, а затем совместно с контурами применяются для улучшения позиции. 

Интересен подход (ссылка на RegioBasedPhotometric), в котором области вокруг границ объекта отслеживаются с использованием цветовых гистограмм, а для того, чтобы контролировать соответствие внутренней части проекции изображению, используются дескрипторы пикселей, основанные на градиенте изображения. Таким образом, алгоритм преодолевает неоднозначности, вызванные симметричностью объекта. 

%При этом, в случае, если изображение зашумлено, то значения дескрипторов, соответствующие одним и тем же точкам, могут сильно изменяться, и это приведёт к зашумлённости функции ошибки  и наличию локальных минимумов.

Предлагаемый в данной статье подход отличается тем, что использует не все пиксели из внутренней части проекции, а только ключевые точки, уверенность в положении которых может быть выше, чем в постоянстве дескрипторов, описывающих произвольные точки на текстуре.

Рассматриваются два способа совмещения. Один из них использует метод ключевых точек для первоначальной оценки положения объекта и инициализации цветового метода, то есть применяет алгоритмы последовательно. Второй способ строит совместную функцкию энергии, основанную как на цветовых гистограммах, так и на особых точках и минимизирует её.

Дальнейшие главы организованы следующим образом: во второй главе метод ключевых точек и метод цветовых гистограмм описаны по отедльности, в третьей главе раскрываются способы их объединения, в четвёртой описывается тестирование и оценка результатов. 