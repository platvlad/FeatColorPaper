\section{Эксперименты}\label{experiments}

Предлагаемый в данной работе метод протестирован с помощью открытого набора
данных OPT\cite{OPT}.

\subsection{Данные, используемые для тестирования}

OPT содержит шесть тестовых предметов, для каждого из которых отснято
по 92 видео с разрешением $1920\times1080$.
В каждом кадре известны истинные позиции объектов.
Предметы делятся на три группы по сложности геометрии 3D-моделей: простая
(Soda, Chest), средняя (Ironman, House) и высокая (Bike, Jet).

Видео записаны в различных условиях освещенности, воспроизводят разнообразные
траектории и скорости движения и поделены на следующие группы тестов:
\begin{itemize}
    \item \term{$x$-$y$ translation} "--- движение по окружности, параллельной
        сенсору камеры.
    \item \term{$z$ translation} (или \term{zoom}) "--- движение вперед и
        назад вдоль оси, перпендикулярной сенсору камеры.
    \item \term{In-plane rotation} "--- вращение вокруг оси, перпендикулярной
        плоскости сенсора камеры.
    \item \term{Out-of-plane rotation} "--- вращение вокруг оси, параллельной
        плоскости сенсора камеры.
    \item \term{Flash light} "--- слабое движение в мигающем свете.
    \item \term{Moving light} "--- слабое движение объекта в свете значительно
        перемещающегося источника освещения.
    \item \term{Free motion} "--- произвольное движение.
\end{itemize}
Группы тестов \term{translation} и \term{rotation} содержат пять различных
уровней скорости.

\subsection{Измерение ошибки}

Ошибка оцененной позиции $\Pose_i$ в кадре $i$ вычисляется как
$
    \delta_i = \avg_{\homv{x} \in \MeshV} \norm{\hat{\Pose_i} \homv{x} - \Pose_i \homv{x}}
$,
где $\MeshV$~--- вершины 3D-модели,
а $\hat{\Pose_i}$ "--- истинная позиция объекта.
Положение объекта считается успешно найденным, если $\delta_i < k d$, где
$d$~--- максимальное из расстояний между всеми парами вершин, а $k$~---
заданный коэффициент ошибки.

Для оценки результатов трекинга строится кривая, каждая точка которой
определяется как процент кадров, на которых была успешно определена позиция
объекта, относительно варьируемого $k$.
Чем больше площадь под данной кривой (AUC), тем выше эффективность
рассматриваемого решения.
В данной работе берутся значения $k$ от 0 до 0.2.
