\section{Эксперименты}\label{experiments}

Предлагаемый в данной работе метод протестирован с помощью открытого набора
данных OPT\cite{OPT}.

\subsection{Данные, используемые для тестирования}

OPT содержит шесть тестовых предметов, для каждого из которых отснято
по 92 видео с разрешением $1920\times1080$.
В каждом кадре известны истинные позиции объектов.
Предметы делятся на три группы по сложности геометрии 3D-моделей: простая
(Soda, Chest), средняя (Ironman, House) и высокая (Bike, Jet).

Видео записаны в различных условиях освещенности, воспроизводят разнообразные
траектории и скорости движения и поделены на следующие группы тестов:
\begin{itemize}
    \item \term{$x$-$y$ translation} "--- движение по окружности, параллельной
        сенсору камеры.
    \item \term{$z$ translation} (или \term{zoom}) "--- движение вперед и
        назад вдоль оси, перпендикулярной сенсору камеры.
    \item \term{In-plane rotation} "--- вращение вокруг оси, перпендикулярной
        плоскости сенсора камеры.
    \item \term{Out-of-plane rotation} "--- вращение вокруг оси, параллельной
        плоскости сенсора камеры.
    \item \term{Flash light} "--- слабое движение в мигающем свете.
    \item \term{Moving light} "--- слабое движение объекта в свете значительно
        перемещающегося источника освещения.
    \item \term{Free motion} "--- произвольное движение.
\end{itemize}
Группы тестов \term{translation} и \term{rotation} содержат пять различных
уровней скорости.

Для тестирования различных модификаций алгоритма собрано подмножество датасета
OPT, в которое отобрано по
четыре видео для каждого объекта и каждой группы, таким образом всего 168
видео.
Для траекторий, для которых доступны видео на разных скоростях, брались видео
со скоростью $4$ (выше средней).
Сравнение итогового алгоритма с аналогами проводится на всём датасете.
Эксперименты проводились на персональном компьютере с процессором Intel Core
i7CPU @ 2.7 GHz.

\subsection{Измерение ошибки}

Ошибка оцененной позиции $\Pose_i$ в кадре $i$ вычисляется как
$
\delta_i = \avg_{\homv{x} \in \MeshV} \norm{\hat{\Pose_i} \homv{x} - \Pose_i
\homv{x}}
$,
где $\MeshV$~--- вершины 3D-модели,
а $\hat{\Pose_i}$ "--- истинная позиция объекта.
Положение объекта считается успешно найденным, если $\delta_i < k d$, где
$d$~--- максимальное из расстояний между всеми парами вершин, а $k$~---
заданный коэффициент ошибки.

Для оценки результатов трекинга строится кривая, каждая точка которой
определяется как процент кадров, на которых была успешно определена позиция
объекта, относительно варьируемого $k$.
Чем больше площадь под данной кривой (AUC), тем выше эффективность
рассматриваемого решения.
В данной работе берутся значения $k$ от 0 до 0.2.

\subsection{Результаты}
Были протестированы метод на основе ключевых точек, метод на основе
распределения цвета и их комбинация.
Также было измерено влияние некоторых модификаций алгоритма на
точность трекинга.
Результаты для каждой группы тестов оцениваются отдельно.

\subsubsection{Комбинирование методов}

В табл.~\ref{tab:combine} представлены результаты трекинга методом на основе
ключевых точек и на основе распределения цвета по отдельности, а также их
комбинацией для разных групп тестов.

\begin{table}[h]
\caption{\label{tab:combine}Результаты комбинирования методов в сравнении с
методами по отдельности}
\begin{center}
\begin{tabular}{|c|c|c|c|c|c|c|c|}
\hline
Метод & Fl & Fm & Ir & Ml & Or & Tr & Zo \\
\hline
Распределение цвета & $6.026$ & $7.757$ & $16.088$ & $5.471$ & $12.963$ &
$10.122$ & $3.743$ \\
\hline
Ключевые точки & $0.717$ & $8.561$ & $15.745$ & $11.248$ & $12.359$ & $12.55$
&$12.219$ \\
\hline
Комбинация & $12.269$ & $10.039$ & $16.932$ & $13.345$ & $12.974$ & $12.958$ &
$12.583$ \\
\hline
\end{tabular}
\end{center}
\end{table}

\Comment{TODO: Заменить в таблице сокращения на полные названия или объяснить
сокращения}

Для всех групп комбинированный метод оказался лучше методов по отдельности.
При этом на группах тестов, где результаты отдельных методов не ниже $10$
(\term{In-plane rotation}, \term{Out-of-plane rotation}, \term{$x$-$y$
translation}), комбинация показывает незначительное улучшение, и её результат
примерно соответствует результату лучшего из отдельных методов.
Это тесты с постоянным освещением, где траекотрией движения является вращение
или перенос.
На них единственным фактором, негативно влияющим на качество трекинга, является высокая скорость, и, как следствие, смазанность изображений.
Оба метода сохраняют работоспособность в таких условиях, и дополнение одного
метода другим незначительно уточняет его работу.

Преимущества комбинированного алгоритма более явно прослеживаются в сложных длятрекинга условиях.
Инициализация алгоритмом ключевых точек может значительно улучшить точность
цветового метода, что можно заметить на примере группы \term{zoom}.
Это можно объяснить тем, что цветовой метод инициализируется вблизи глобального минимума функции энергии, и даже если оптимизация сойдётся к локальному
минимуму, то результат будет недалёк от истинной позиции.

В группе тестов с мигающим светом (\term{Flash light}) оба отдельных метода
испытывают сложности, однако в комбинации удерживают друг друга вблизи истиннойпозиции.
В случае получения неправильной позиции методом ключевых точек цветовой
алгоритм может быть проинициализирован экстраполяцией движения, что также
поддерживает общую устойчивость алгоритма.


\Comment{TODO: Объяснить плохой результат slsqp на zoom?}
\Comment{TODO: Лучше объяснить странный результат на FL}

\subsubsection{Организация гистограмм}

В данной работе вводится подход к организации гистограмм, ограничивающий их
количество константой.
В то же время это потенциально может ухудшить точность метода в сравнении с
подходом, в котором для каждой вершины используется пара гистограмм.
В табл.~\ref{tab:full_hist} сравниваются два способа реализации
комбинированного алгоритма: в первом из них гистограммы организованы согласно
описанию в главе~\ref{tracking}.
Во втором испольузется подход, описанный в~\cite{Tjaden2017}
и~\cite{Tjaden2018}: пара гистограмм строится для каждой вершины и действует в
окрестности её проекции, если эта проекция оказалась вблизи контура.
При тестировании мы использовали гистограммы всех вершин, спроецированных рядом
с контуром, в отличие от~\cite{Tjaden2017} и~\cite{Tjaden2018}, где выбиралась
только часть таких вершин.
Это позволяет обновлять информацию в большем числе гистограмм и избавляет от
рисков, связанных с неправильным их выбором, хотя и ухудшает
производительность.

\Comment{Результаты пока только на объекте Bike}

\begin{table}[h]
\caption{\label{tab:full_hist}Влияние организации гистограмм на точность}
\begin{center}
\begin{tabular}{|c|c|c|c|c|c|c|c|}
\hline
Метод & Fl & Fm & Ir & Ml & Or & Tr & Zo \\
\hline
32 пары гистограмм & $13.46$ & $7.027$ & $17.818$ & $15.51$ & $8.913$ & $12.09$
& $12.397$ \\
\hline
Пара гистограмм на каждую вершину & $4.456$ & $10.034$ & $15.458$ & $13.703$ &
$12.279$ & $10.141$ & $15.183$ \\
\hline
\end{tabular}
\end{center}
\end{table}

Можно заметить, что на большинстве паттернов результаты не ухудшились, что
подтверждает состоятельность выбранного способа организации гистограмм.

\subsubsection{Влияние пересчёта 3D-прообразов ключевых точек}

Позиция, полученная цветовым методом, используется для задания 3D-прообразов
ключевых точек, появившихся на текущем кадре впервые.
Вместе с тем, она используется и для коррекции старых точек, если эта коррекция
уменьшает ошибку репроекции.

Если не обновлять 3D-прообразы старых ключевых точек, то они вычисляются
единожды при первом появлении ключевой точки на видео.
Даже если этот прообраз был найден неточно, он в дальнейшем не обновляется.

В табл.~\ref{tab:reprojection} представлены результаты трекинга с коррекцией
старых 3D-позиций и того же алгоритма, в котором 3D-позиции вычисляются
только при первом появлении ключевой точки и далее не пересчитываются.

\begin{table}[h]
\caption{\label{tab:reprojection}Влияние пересчёта 3D-прообразов}
\begin{center}
\begin{tabular}{|c|c|c|c|c|c|c|c|}
\hline
Метод & Fl & Fm & Ir & Ml & Or & Tr & Zo \\
\hline
С коррекцией старых точек & $12.269$ & $10.039$ & $16.932$ & $13.345$ &
$12.974$ &
$12.958$ & $12.583$ \\
\hline
Вычисление 3D-позиций при первом появлении & $11.209$ & $8.651$ & $16.649$ &
$13.698$ & $12.788$ & $12.975$ &
$13.59$ \\
\hline
\end{tabular}
\end{center}
\end{table}

Коррекция 3D-позиций в ходе трекинга улучшает точность на группе \term{flash
light}, где алгоритм ключевых точек сам по себе работает плохо.
Также заметно улучшение на группе \term{free motion}, что можно объяснить
большим разбросом 3D-прообразов одной и той же точки, вычисленных на разных
кадрах.
Это увеличивает вероятность того, что первоначально вычисленный 3D-прообраз
оказался недостаточно точен, но можно будет найти лучший по следующим кадрам.

На группе \term{zoom} цветовой трекинг менее точен, чем трекинг на ключевых
точках.
В ходе трекинга точность может ухудшаться, поэтому и обновление позиции по
менее точным кадрам несколько ухудшает результаты.
На остальных группах пересчёт позиций не улучшает качество трекинга, либо
улучшает его незначительно.

\Comment{Объяснения пока очень предположительные}

\subsubsection{Сравнение с аналогами}

В табл.~\ref{tab:analogues} представлено сравнение предлагаемого алгоритма с
алгоритмами на основе распределения цвета (PWP3D, RBOT) и комбинированными
алгоритмами (Bugaev et.al., Zhong et.al.).

\begin{itemize}
\item PWP3D "--- один из первых цветовых алгоритмов, в котором использовалась
одна пара гистограмм для всего изображения.
\item RBOT "--- цветовой алгоритм с одной парой гистограмм для каждой вершины
объекта.
\item Bugaev et. al. "--- алгоритм, в котором совмещались трекинг на ключевых
точках с методом, основанным на контурах. Ключевые точки также использовались
для инициализации трекинга, а затем --- для задания ограничений при оптимизации
контурной функции ошибки.
\item Zhong et.al. "--- алгоритм, в котором отслеживается распределение цвета в
окрестностях контура и граидентные дескрипторы на внутренней части переднего
плана.
\end{itemize}

\begin{table}[h]
\caption{\label{tab:analogues}Сравнение с другими цветовыми и комбинированными
алгоритмами}
\begin{center}
\begin{tabular}{|c|c|c|c|c|c|c|c|}
\hline
Объект & Bike & Chest & House & Ironman & Jet & Soda & Среднее \\
\hline
PWP3D~\cite{PWP3D} & $5.358$ & $5.551$ & $3.575$ &
$3.915$ & $5.813$ & $5.87$ & $5.014$ \\
\hline
RBOT~\cite{Tjaden2018} & $11.903$ & $11.764$ & $10.15$ &
$11.986$ & $13.217$ & $8.861$ & $11.314$ \\
\hline
Bugaev et.al.~\cite{Bugaev_2018_ECCV} & $12.55$ & $14.97$ & $14.48$ &
$14.71$ & $17.17$ & $14.85$ & $14.79$ \\
\hline
Zhong et.al.~\cite{Zhong2020} & $12.831$ & $12.24$ & $13.613$ &
$11.214$ & $15.441$ & $9.012$ & $12.392$ \\
\hline
Наш алгоритм & $11.648$ & $15.522$ & $16.607$ & $11.556$ & $13.966$ & $14.104$
& $13.901$ \\
\hline
\end{tabular}
\end{center}
\end{table}

За счёт локальности RBOT и наш алгоритм работают значительно лучше PWP3D. 
Можно заметить, что наш алгоритм точнее RBOT на объектах простой геометрии
(Chest, Soda) и примерно соответствует ему по точности на объектах сложной
геометрии (Bike, Jet).
На объектах сложной геометрии RBOT может иметь преимущество, так как использует
большее количество гистограмм.
Ключевые точки на таких объектах использовать сложнее, так как при большом
количестве треугольников они могут перекрывать друг друга, из-за чего точки на
перекрытых треугольниках отфильтровываются.

На объектах более простой геометрии, предложенный комбинированный метод
значительно превосходит RBOT.
Преимущества комбинирования явно прослеживаются на объекте Soda. 
Этот объект симметричен относительно оси вращения, и цветовой алгоритм
испытывает трудности при его трекинге.
Использование ключевых точек помогает разрешать неоднозначности, связанные с
одинаковым контуром при разных углах поворота объекта вокруг своей оси.

\Comment{TODO: связь сложности геометрии с точностью трекинга?}
\Comment{TODO: больше алгоритмов для сравнения?}
\Comment{TODO: больше аналитики}
