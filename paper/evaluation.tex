\section{Эксперименты}\label{experiments}

Предлагаемый в данной работе метод протестирован с помощью открытого набора
данных OPT\cite{OPT}.

\subsection{Данные, используемые для тестирования}

OPT содержит шесть тестовых предметов, для каждого из которых отснято
по 92 видео с разрешением $1920\times1080$.
В каждом кадре известны истинные позиции объектов.
Предметы делятся на три группы по сложности геометрии 3D-моделей: простая
(Soda, Chest), средняя (Ironman, House) и высокая (Bike, Jet).

Видео записаны в различных условиях освещенности, воспроизводят разнообразные
траектории и скорости движения и поделены на следующие группы тестов:
\begin{itemize}
    \item \term{$x$-$y$ translation} "--- движение по окружности, параллельной
        сенсору камеры.
    \item \term{$z$ translation} (или \term{zoom}) "--- движение вперед и
        назад вдоль оси, перпендикулярной сенсору камеры.
    \item \term{In-plane rotation} "--- вращение вокруг оси, перпендикулярной
        плоскости сенсора камеры.
    \item \term{Out-of-plane rotation} "--- вращение вокруг оси, параллельной
        плоскости сенсора камеры.
    \item \term{Flash light} "--- слабое движение в мигающем свете.
    \item \term{Moving light} "--- слабое движение объекта в свете значительно
        перемещающегося источника освещения.
    \item \term{Free motion} "--- произвольное движение.
\end{itemize}
Группы тестов \term{translation} и \term{rotation} содержат пять различных
уровней скорости.

Для тестирования собрано подмножество датасета OPT, в которое отобрано по
одному видео для каждого объекта и каждой группы, таким образом всего 42 видео.
Эксперименты проводились на персональном компьютере с процессором Intel Core i7CPU @ 2.7 GHz.

\subsection{Измерение ошибки}

Ошибка оцененной позиции $\Pose_i$ в кадре $i$ вычисляется как
$
\delta_i = \avg_{\homv{x} \in \MeshV} \norm{\hat{\Pose_i} \homv{x} - \Pose_i
\homv{x}}
$,
где $\MeshV$~--- вершины 3D-модели,
а $\hat{\Pose_i}$ "--- истинная позиция объекта.
Положение объекта считается успешно найденным, если $\delta_i < k d$, где
$d$~--- максимальное из расстояний между всеми парами вершин, а $k$~---
заданный коэффициент ошибки.

Для оценки результатов трекинга строится кривая, каждая точка которой
определяется как процент кадров, на которых была успешно определена позиция
объекта, относительно варьируемого $k$.
Чем больше площадь под данной кривой (AUC), тем выше эффективность
рассматриваемого решения.
В данной работе берутся значения $k$ от 0 до 0.2.

\subsection{Результаты}
Были протестированы метод на основе ключевых точек, метод на основе
распределения цвета и их комбинация.
Также было измерено влияние некоторых модификаций цветового алгоритма на
точность трекинга.
Результаты для каждой группы тестов оцениваются отдельно.

\subsubsection{Комбинирование методов}

В табл.~\ref{tab:combine} представлены результаты трекинга методом на основе
ключевых точек и на основе распределения цвета по отдельности, а также их
комбинацией для разных групп тестов.

\begin{table}[h]
\caption{\label{tab:combine}Результаты комбинирования методов в сравнении с
методами по отдельности}
\begin{center}
\begin{tabular}{|c|c|c|c|c|c|c|c|}
\hline
Метод & Fl & Fm & Ir & Ml & Or & Tr & Zo \\
\hline
Распределение цвета & $9.931$ & $7.144$ & $14.141$ & $11.577$ & $15.627$ &
$13.94$ & $10.85$ \\
\hline
Ключевые точки & $0.528$ & $8.257$ & $16.542$ & $12.165$ & $14.26$ & $13.576$ &
$15.163$ \\
\hline
Комбинация & $11.561$ & $10.016$ & $18.371$ & $15.05$ & $15.082$ & $15.037$ &
$16.418$ \\
\hline
\end{tabular}
\end{center}
\end{table}

\Comment{TODO: Заменить в таблице сокращения на полные названия или объяснить
сокращения}

Для всех групп, кроме \term{Out-of-plane rotation}, комбинированный метод
оказался лучше методов по отдельности.
При этом на группах тестов, где результаты отдельных методов не ниже $12$
(\term{In-plane rotation}, \term{Out-of-plane rotation}, \term{$x$-$y$
translation}), результат комбинации превосходит лучший из методов не более, чем
на $2$ пункта.

В случае явного падения точности метода ключевых точек, а именно, в группе
\term{Flash light}, цветовой трекинг инициализируется экстраполяцией движения,
поэтому результат комбинации на этой группе не уступает результату цветового
алгоритма.

Использование ключевых точек также может значительно улучшить точность
цветового метода, что можно заметить на примере группы \term{zoom}.
Это можно объяснить тем, что результат метода ключевых точек инициализирует
оптимизацию вблизи глобального минимума функции энергии, и даже если
оптимизация сойдётся к локальному минимуму, то результат будет недалёк от
истинной позиции.

\Comment{TODO: Объяснить плохой результат slsqp на zoom?}
\Comment{TODO: Объяснить, почему комбинация хуже на Tr?}

\subsubsection{Влияние эквализации гистограмм по яркости}

На вход цветового алгоритма изображение подаётся обработанным: оно переводится
в формат HSV, и затем проводится эквализация гистограммы канала яркости.
При комбинировании алгоритмов выделение ключевых точек происходит на исходном
изображении, а цветовые гистограммы строятся и обновляются на обработанном.
В табл.~\ref{tab:hsv_slsqp} представлены результаты этой модификации цветового
алгоритма по сравнению с использованием RGB-изображений.

\begin{table}[h]
\caption{\label{tab:hsv_slsqp}Влияние формата изображения на цветовой алгоритм}\begin{center}
\begin{tabular}{|c|c|c|c|c|c|c|c|}
\hline
Формат изображений & Fl & Fm & Ir & Ml & Or & Tr & Zo \\
\hline
RGB & $6.604$ & $4.281$ & $12.881$ & $3.608$ & $14.36$ & $12.335$ & $10.39$ \\
\hline
HSV, эквализация гистограмм яркости & $9.931$ & $7.144$ & $14.141$ & $11.577$ &
$15.627$ & $13.94$ & $10.85$ \\
\hline
\end{tabular}
\end{center}
\end{table}

Основным результатом эквализации гистограмм изображений по яркости стало
улучшение работы цветового алгоритма при меняющемся освещении (группы тестов
\term{Flash light} и \term{Moving light}).
При использовании RGB-изображений резкое изменение освещения меняет значения
цветов, и на вновь поступившем кадре часто встречаются цвета, по которым не
собрано ещё достаточно статистики.
Эквализация гистограмм в значительной степени нивелирует влияние освещения на
изображение сцены, и цвета одних и тех же точек на разных изображениях меняются
гораздо меньше.

Операция эквализации гистограмм зависит от изображения, на котором она
проводится.
Таким образом, на разных изображениях проводятся разные преобразования. 
Это может сделать собранную статистику по цветам менее пригодной для анализа
нового изображения, что может ухудшить качество цветового трекинга.
Тем не менее, на проведённых тестах обработка изображений в среднем не ухудшила
точность даже на тех видео, где освещение было постоянным.

\Comment{TODO: перенести часть текста выше в технические главы?}

Эквализация гистограмм гораздо меньше влияет на комбинацию алгоритмов, как
видно из табл.~\ref{tab:hsv_lkt_init}.
Результаты на тестах с меняющимся освещением всё ещё улучшаются, но лишь в
пределах $2$ пунктов.
Это объясняется тем, что метод ключевых точек инициализирует функцию энергии
цветового алгоритма вблизи истинной позиции, и оптимизация функции энергии
остаётся вблизи неё.

\begin{table}[h]
\caption{\label{tab:hsv_lkt_init}Влияние формата изображения на комбинированный
алгоритм}
\begin{center}
\begin{tabular}{|c|c|c|c|c|c|c|c|}
\hline
Формат изображений & Fl & Fm & Ir & Ml & Or & Tr & Zo \\
\hline
RGB & $10.553$ & $10.318$ & $18.36$ & $13.171$ & $15.924$ & $15.382$ & $16.358$
\\
\hline
HSV, эквализация гистограмм яркости & $11.561$ & $10.016$ & $18.371$ & $15.05$
& $15.082$ & $15.037$ & $16.418$ \\
\hline
\end{tabular}
\end{center}
\end{table}

На тех видео, где освещение было постоянным, результаты на обработанных и
необработанных изображениях мало отличаются.
В некоторых случаях результаты трекинга на RGB-изображениях лучше, так как
эквализация гистограмм может быть бесполезной при постоянном освещении.
Тем не менее, при изменении освещения эквализация гистограмм выигрывает даже в
случае инициализации цветового трекинга алгоритмом ключевых точек.


