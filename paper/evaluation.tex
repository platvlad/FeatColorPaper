\section{Эксперименты}\label{experiments}

Предлагаемый в данной работе метод протестирован с помощью открытого набора
данных OPT\cite{OPT}.

\subsection{Данные, используемые для тестирования}

OPT содержит шесть тестовых предметов, для каждого из которых отснято
по 92 видео с разрешением $1920\times1080$.
В каждом кадре известны истинные позиции объектов.
Предметы делятся на три группы по сложности геометрии 3D-моделей: простая
(Soda, Chest), средняя (Ironman, House) и высокая (Bike, Jet).

Видео записаны в различных условиях освещенности, воспроизводят разнообразные
траектории и скорости движения и поделены на следующие группы тестов:
\begin{itemize}
    \item \term{$x$-$y$ translation} "--- движение по окружности, параллельной
        сенсору камеры.
    \item \term{$z$ translation} (или \term{zoom}) "--- движение вперед и
        назад вдоль оси, перпендикулярной сенсору камеры.
    \item \term{In-plane rotation} "--- вращение вокруг оси, перпендикулярной
        плоскости сенсора камеры.
    \item \term{Out-of-plane rotation} "--- вращение вокруг оси, параллельной
        плоскости сенсора камеры.
    \item \term{Flash light} "--- слабое движение в мигающем свете.
    \item \term{Moving light} "--- слабое движение объекта в свете значительно
        перемещающегося источника освещения.
    \item \term{Free motion} "--- произвольное движение.
\end{itemize}
Группы тестов \term{translation} и \term{rotation} содержат пять различных
уровней скорости.

Для тестирования различных модификаций алгоритма собрано подмножество датасета
OPT, в которое отобрано по
четыре видео для каждого объекта и каждой группы, таким образом всего 168
видео.
Для траекторий, для которых доступны видео на разных скоростях, брались видео
со скоростью $4$ (выше средней).
Сравнение итогового алгоритма с аналогами проводится на всём датасете.
Эксперименты проводились на персональном компьютере с процессором Intel Core
i7CPU @ 2.7 GHz.

\subsection{Измерение ошибки}

Ошибка оцененной позиции $\Pose_i$ в кадре $i$ вычисляется как
$
\delta_i = \avg_{\homv{x} \in \MeshV} \norm{\hat{\Pose_i} \homv{x} - \Pose_i
\homv{x}}
$,
где $\MeshV$~--- вершины 3D-модели,
а $\hat{\Pose_i}$ "--- истинная позиция объекта.
Положение объекта считается успешно найденным, если $\delta_i < k d$, где
$d$~--- максимальное из расстояний между всеми парами вершин, а $k$~---
заданный коэффициент ошибки.

Для оценки результатов трекинга строится кривая, каждая точка которой
определяется как процент кадров, на которых была успешно определена позиция
объекта, относительно варьируемого $k$.
Чем больше площадь под данной кривой (AUC), тем выше эффективность
рассматриваемого решения.
В данной работе берутся значения $k$ от 0 до 0.2.

\subsection{Результаты}
Были протестированы метод на основе ключевых точек, метод на основе
распределения цвета и их комбинация.
Также было измерено влияние некоторых модификаций алгоритма на
точность трекинга.
Результаты для каждой группы тестов оцениваются отдельно.

\subsubsection{Комбинирование методов}

В табл.~\ref{tab:combine} представлены результаты трекинга методом на основе
ключевых точек и на основе распределения цвета по отдельности, а также их
комбинацией для разных групп тестов.

\begin{table}[h]
\caption{\label{tab:combine}Результаты комбинирования методов в сравнении с
методами по отдельности}
\begin{center}
\begin{tabular}{|c|c|c|c|c|c|c|c|}
\hline
Метод & Fl & Fm & Ir & Ml & Or & Tr & Zo \\
\hline
Распределение цвета & $6.026$ & $7.757$ & $16.088$ & $5.471$ & $12.963$ &
$10.122$ & $3.743$ \\
\hline
Ключевые точки & $0.717$ & $8.561$ & $15.745$ & $11.248$ & $12.359$ & $12.55$ &$12.219$ \\
\hline
Комбинация & $12.269$ & $10.039$ & $16.932$ & $13.345$ & $12.974$ & $12.958$ &
$12.583$ \\
\hline
\end{tabular}
\end{center}
\end{table}

\Comment{TODO: Заменить в таблице сокращения на полные названия или объяснить
сокращения}

Для всех групп комбинированный метод оказался лучше методов по отдельности.
При этом на группах тестов, где результаты отдельных методов не ниже $10$
(\term{In-plane rotation}, \term{Out-of-plane rotation}, \term{$x$-$y$
translation}), комбинация показывает незначительное улучшение, и её результат
примерно соответствует результату лучшего из отдельных методов.
Это тесты с постоянным освещением, где траекотрией движения является вращение
или перенос.
На них единственным фактором, негативно влияющим на качество трекинга, является
высокая скорость, и, как следствие, смазанность изображений.
Оба метода сохраняют работоспособность в таких условиях, и дополнение одного
метода другим незначительно уточняет его работу.

Преимущества комбинированного алгоритма более явно прослеживаются в сложных для
трекинга условиях.
Инициализация алгоритмом ключевых точек может значительно улучшить точность
цветового метода, что можно заметить на примере группы \term{zoom}.
Это можно объяснить тем, что цветовой метод инициализируется вблизи глобального
минимума функции энергии, и даже если оптимизация сойдётся к локальному
минимуму, то результат будет недалёк от истинной позиции.

В группе тестов с мигающим светом (\term{Flash light}) оба отдельных метода
испытывают сложности, однако в комбинации удерживают друг друга вблизи истинной
позиции.
В случае получения неправильной позиции методом ключевых точек цветовой
алгоритм может быть проинициализирован экстраполяцией движения, что также
поддерживает общую устойчивость алгоритма.


\Comment{TODO: Объяснить плохой результат slsqp на zoom?}
\Comment{TODO: Лучше объяснить странный результат на FL}

\subsubsection{Организация гистограмм}

\subsubsection{Влияние пересчёта 3D-прообразов ключевых точек}

Позиция, полученная цветовым методом, используется для задания 3D-прообразов
ключевых точек, появившихся на текущем кадре впервые.
Вместе с тем, она используется и для коррекции старых точек, если новая позиция
оказалась лучше.
В табл.~\ref{tab:reprojection} представлены результаты трекинга с коррекцией
старых 3D-позиций и того же алгоритма, в котором 3D-позиции вычисляются
единожды при первом появлении ключевой точки и далее не пересчитываются.

\begin{table}[h]
\caption{\label{tab:reprojection}Влияние пересчёта 3D-прообразов}
\begin{center}
\begin{tabular}{|c|c|c|c|c|c|c|c|}
\hline
Метод & Fl & Fm & Ir & Ml & Or & Tr & Zo \\
\hline
С коррекцией старых точек & $12.269$ & $10.039$ & $16.932$ & $13.345$ &
$12.974$ &
$12.958$ & $12.583$ \\
\hline
Вычисление 3D-позиций при первом появлении & $11.209$ & $8.651$ & $16.649$ &
$13.698$ & $12.788$ & $12.975$ &
$13.59$ \\
\hline
\end{tabular}
\end{center}
\end{table}

Коррекция 3D-позиций в ходе трекинга улучшает точность на группе \term{flash
light}, где алгоритм ключевых точек сам по себе работает плохо.
Также заметно улучшение на группе \term{free motion}, что можно объяснить
большим разбросом 3D-прообразов одной и той же точки, вычисленных на разных
кадрах.
Это увеличивает вероятность того, что первоначально вычисленный 3D-прообраз
оказался недостаточно точен, но можно будет найти лучший по следующим кадрам.

На группе \term{zoom} цветовой трекинг менее точен, чем трекинг на ключевых
точках.
В ходе трекинга точность может ухудшаться, поэтому и обновление позиции по
менее точным кадрам несколько ухудшает результаты.
На остальных группах пересчёт позиций не улучшает качество трекинга, либо
улучшает его незначительно.
