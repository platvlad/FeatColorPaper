\section{Связанные работы}

\Comment{TODO: перенести подробные обзоры из введения и дописать.}

Особняком стоят методы, использующие для трекинга информацию о цветах объекта.
Проекция модели в некоторой позиции разбивает плоскость изображения на две
области, называемые далее передний план и фон.
В кадрах, для которых истинная позиция объекта известна, передний план
соответствует объекту, что позволяет строить гистограммы распределения цветов
поверхности объекта и фона.
Эти гистограммы затем можно использовать для вычисления вероятности того, что в
заданной позиции проекция модели верно отделяет объект от фона на кадре.
Искомой позицией считается та, в которой такая вероятность максимальна.
Трекинг в таком случае производится путем численной оптимизации функции
энергии, соответствующей данной вероятности.

Первой работой, использующей описанные принципы для 3D-трекинга,
был метод PWP3D\cite{PWP3D}.

\Comment{TODO: больше подробностей про особенности PWP3D.}

В \cite{Tjaden2017} предложено использовать не одну пару гистограмм для модели,
а по две гистограммы для каждой вершины.
В таком случае область действия каждой гистограммы сокращается до круга
определённого радиуса вокруг проекции соответствующей вершины.
Это позволяет учитывать информацию о цветах локально.
Такое количество данных может быть избыточно.
В \cite{RegionPhotometric} предлагается способ ограничить это количество, но он
основан на разбиении пространства изображения на секторы, и  в случае поворота
объекта отдельная гистограмма фиксирует цвета с разных частей объекта, тогда
как в \cite{Tjaden2017} гистограммы привязаны к определённым вершинам и хранят
информацию только об окрестностях их проекций.

\Comment{TODO: причесать и добавить подробностей.}
