\section{Связанные работы}\label{related-work}

Подробное описание различных методов 3D-трекинга можно найти в обзорах
\cite{LepetitSurvey,MarchandSurvey}.
В данном разделе мы кратко разберем основные подходы и
подробно остановимся на методах, основанных на информации о распределении
цветов.

Множество методов 3D-трекинга
\cite{Hinterstoisser2007,Vacchetti2004,Lourakis2013,Pauwels2013}
основано на использовании информации о локальных
особенностях \cite{AKAZE,SIFT,ShiAndTomasi,TomasiAndKanade,SIFT,PyrLK}.
Предполагается, что особая точка должна соответствовать определенной точке на 
поверхности объекта. Таким образом, может быть составлен набор 2D-3D 
соответствий, по которым в дальнейшем вычисляется
позиция объекта. Такие методы устойчивы к частичным перекрытиям объекта, к
сильно-текстурированному фону и хорошо справляются с отслеживанием
симметричныхобъектов. Главным недостатком подходов на основе локальных
особенностей
является
необходимое условие наличия у объекта неоднородной текстуры, на которой можно
было бы задетектировать достаточно большое количество особых точек. Поэтому
такие методы показывают неудовлетворительные результаты при отслеживании
слабо-текстурированных объектов.

Другие подходы используют предположение о том, что изображение объекта хорошо
отделяется от фона. Проекция объекта разделяет плоскость изображения на две
области. Одна область соответствует изображению объекта и называется
\term{передним планом}. Оставшуюся область изоражения будем называть
\term{фоном}. Многие методы использует предположение о том, что границе
данных областей, называемой \term{контуром}, соответствуют области изображения
с
высоким значением градиента "--- \term{границы} на изображении \cite{CANNY}. В
работах
\cite{RAPID,Marchand2003,Choi2012,Marchand2006,Klein2006,SeoHinterstoisser2014,WangZhong2015,Damen2012,VacchettiEdges2004}
вычисляются 2D-3D соответствия между границами на изображении и ближайшими
точками спроецированного контура объекта. Полученные соответствия используются
для вычисления позиции. Другие подходы
\cite{WangZhong2017,Marchand2001,Bugaev_2018_ECCV} вычисляют позицию объекта в
процессе оптимизации функций энергии, максимизирующей соответствие контуров
объекта границам изображения. Все методы на основе границ могут устойчиво
работать на слабо-текстурированных объектах, но при этом они подвержены
проблеме
попадания в локальные оптимумы при наличии сильно-текстурированного фона.
Такжетакие методы не достаточно приспособлены для трекинга симметричных
объектов.

Подходы, основанные только на соответствии контуров объекта границам
изображения,
не используют информацию о внутренней области объекта, которая может
существенно
увеличивать устойчивость и точность. В работах
\cite{VacchettiEdges2004,ChoiFeaturesAndEdges,Bugaev_2018_ECCV} предлагается
комбинировать подходы на основе границ и подходы на основе ключевых точек.
Метод
\cite{Bugaev_2018_ECCV} использует локальные особенности для повышения
устойчивости алгоритма определения позиции объекта на основе оптимизации
функции
энергии контуров. Такой подход эффективен при отслеживании симметричных
объектов, а также лучше работает в сложных условиях с сильно-текстурированным
фоном, чем подходы основанный только на границах. Однако симметричные объекты
соднородной тестурой и, как следствие, малым количеством локальных
особенностейна текстуре, могут оказаться проблемой для рассматриваемого
метода.Также
сильно
текстурированный фон в сочетании со слабой текстурированностью объекта может
привести к неустойчивому трекингу.

Другим подходом, основанном на отделении переднего плана от фона,
является использование распределения цветов в этих областях. Предполагается,
что
статистика цветов объекта заметно отличается от статистики цветов фона.
Некоторые методы \cite{SeoHinterstoisser2014,WangZhong2015,Zhong2018}
используют
информацию о цвете для улучшения подходов на основе контуров. Например, в
работах \cite{SeoHinterstoisser2014,WangZhong2015} распределение цветов в
окрестности граничной точки изображения используется для отсеивания ложных
2D-3D
соответствий между границами на изображении и контурами объекта.

В последние годы было предложно множество подходов, которые предлагают
непосредственно вычислять позицию объекта на основе сегментации переднего
планаи фона. В кадрах, для которых истинная позиция объекта известна, передний
план
соответствует объекту, а фон всему остальному изображению, что позволяет
строить гистограммы распределения цветов поверхности объекта и фона.
Эти гистограммы затем можно использовать для вычисления вероятности того, что
взаданной позиции проекция модели верно отделяет объект от фона на кадре.
Искомой позицией считается та, в которой такая вероятность максимальна.
Трекинг в таком случае производится путем численной оптимизации функции
энергии, соответствующей данной вероятности.
Первой работой, использующей описанные принципы для 3D-трекинга,
был метод PWP3D\cite{PWP3D}. Данный подход хорошо подходит для трекинга
слабо-текстурированных объектов, а также устойчив к сильно-текстурированному
фону. Однако метод имеет несколько недостатков.
Во-первых, в процессе оптимизации используется низкоэффективный метод
градиентоного спуска с фиксированным шагом. Во-вторых, В PWP3D для задания
распределений цветов фона и объекта используется глобальная модель, состоящая
из
одной гистограммы для фона и одной гистограммы для объекта. Такой подход не
учитывает локальность распределения цвета, а также неустойчив к изменению фона
и
перекрытиям объекта.

Поэтому, многие последующие работы были сосредоточены на повышении
эффективности
метода PWP3D.
В работе \cite{Hexner2016} было предложено использовать локальные гистограммы
цветов для регионов вокруг 2D-точек, соответствующих контуру, вместо
глобальной модели распределения цветов.
В таком случае область действия каждой гистограммы сокращается до круга
определённого радиуса вокруг точки контура.
Это позволяет учитывать информацию о цветах локально.
Точки берутся именно на контуре, так как в этом случае их окрестность
захватывает
как область переднего плана, так и фона.
Из-за этого область, на которой собирается статистика цветов, представляет
собой полосу вокруг границы объекта.
Оптимизация направлена на то, чтобы контур максимизировал
различие цветов на переднем плане и на фоне.
Гистограммы для окрестностей точек контура помогают использовать цветовую
информацию локально, но не позволяют накапливать информацию в ходе трекинга,
так как привязаны только к точке контура на данном изображении.
В работе \cite{Tjaden2017} предложено использовать по две гистограммы для
каждой вершины модели.
Область действия таких гистограмм "--- окрестность проекции вершины.
На каждой итерации используются гистограммы тех вершин, проекция которых
близка к контуру.
Поэтому, как и в~\cite{Hexner2016}, рассматриваются цвета точек на полосе
вокруг границы объекта.
Однако привязка гистограмм к вершинам модели не только делает их локальными,
но и даёт возможность собирать цветовую статистику в ходе трекинга.
При этом такое количество гистограмм в случае большого числа вершин приводит к
существенному потреблению памяти и снижению скорости работы алгоритма.
Хотя в~\cite{Tjaden2017} предлагается на каждом кадре обновлять не более 100
гистограмм, случайно взятых вблизи контура, их выбор нетривиален и может влиять
на качество трекинга\Comment{Предположение, может быть неуместно}.
 В работе \cite{Tjaden2018} было предложено использовать
эффективную стратегию оптимизации на основе алгоритма Гаусса "--- Ньютона. Это
позволило избежать необходимости ручного
подбора шага, а также ускорить сходимость метода. 
\Comment{Нужно подробнее рассказать о том, как используются эти две
гистограммы для вершины, ведь некоторые из них далеко от фона "--- это
непонятно.}
\Comment{TODO Пояснить про полосу около границы и внутренность.}
В \cite{Zhong2018} предлагается способ ограничения числа гистограмм
путем разбиения пространства изображения на секторы. В случае поворота объекта
отдельная гистограмма фиксирует цвета с разных частей
объекта, тогда как в \cite{Tjaden2017} гистограммы привязаны к определённым
вершинам и хранят информацию только об окрестностях их проекций.
Области вокруг границ объекта отслеживаются с использованием цветовых
гистограмм, а для того, чтобы контролировать соответствие внутренней части
проекции изображению, используются дескрипторы пикселей, основанные на
градиенте изображения. В результате, данный метод комбинирует информацию о
статистике цветов объекта и информацию о локальных градиентах текстуры
объекта,что позволяет преодолевать неоднозначности, вызванные симметричностью
объекта.


В нашей работе предлагается метод выбора локальных гистограмм, позволяющий
существенно сократить их число. 
Подобно методам~\cite{Hexner2016,Tjaden2017,Tjaden2018}
и~\cite{Zhong2018}, статистика собирается на полосе вокруг контура объекта.
Поверхность объекта разделяется на несколько областей, а её проекция
соответственно разбивает на области контур.
Цвета точек окрестности контура попадают в гистограмму, соответствующую
ближайшей точке на контуре.
Разделение на области самого объекта позволяет накапливать статистику по этим
областям в ходе трекинга.
При этом, вне зависимости от поворотов объекта, гистограммы будут сохранять
данные о цвете определённых участков поверхности.
\Comment{TODO Подробнее о методе выбора гистограмм}
В результате комбинирования информации о статистике цветов с информацией об
особых точках на текстуре объета, наш метод преодолевает неоднозначности при
трекинге симметричных объектов.

В работе \cite{Zhao2014} также было предложено использовать фильтр частиц,
позволяющий уменьшить вероятность попадания в локальный оптимум в процессе
оптимизации. В нашей работе для преодоления проблемы попадания в локальный
оптимум предлагается использование априорной оценки позиции на основе особых
точек.
\Comment{TODO Подробнее о повышении стабильности оптимизации за счет
использования особых точек}
