\section{Связанные работы}

\Comment{TODO: перенести подробные обзоры из введения и дописать.}

Особняком стоят методы, использующие для трекинга информацию о цветах объекта.
Проекция модели в некоторой позиции разбивает плоскость изображения на две
области, называемые далее передний план и фон.
В кадрах, для которых истинная позиция объекта известна, передний план
соответствует объекту, а фон всему за исключением объекта, что позволяет
строить гистограммы распределения цветов поверхности объекта и фона.
Эти гистограммы затем можно использовать для вычисления вероятности того, что в
заданной позиции проекция модели верно отделяет объект от фона на кадре.
Искомой позицией считается та, в которой такая вероятность максимальна.
Трекинг в таком случае производится путем численной оптимизации функции
энергии, соответствующей данной вероятности.

Первой работой, использующей описанные принципы для 3D-трекинга,
был метод PWP3D\cite{PWP3D}.
В нем для задания распределений цветов фона и объекта используется по одной
гистограмме.
Это является слабым местом данного алгоритма, поскольку не учитывает
локальность распределения цвета, а также неустойчиво к изменению фона
и перекрытиям объекта.

В работе \cite{Tjaden2017} предложено использовать не одну пару гистограмм для
всего алгоритма, а по две гистограммы для каждой вершины модели.
В таком случае область действия каждой гистограммы сокращается до круга
определённого радиуса вокруг проекции соответствующей вершины.
Это позволяет учитывать информацию о цветах локально.
Однако такое количество гистограмм в случае большого числа вершин приводит к
существенному потреблению памяти и снижению скорости работы алгоритма.
\Comment{Нужно подробнее рассказать о том, как используются эти две
гистограммы для вершины, ведь некоторые из них далеко от фона "--- это
непонятно.}

В \cite{RegionPhotometric} предлагается способ ограничить число гистограм
путем разбиения пространства изображения на секторы.
В случае поворота объекта отдельная гистограмма фиксирует цвета с разных частей
объекта, тогда как в \cite{Tjaden2017} гистограммы привязаны к определённым
вершинам и хранят информацию только об окрестностях их проекций.
Области вокруг границ объекта отслеживаются с использованием цветовых
гистограмм, а для того, чтобы контролировать соответствие внутренней части
проекции изображению, используются дескрипторы пикселей, основанные на
градиенте изображения.
Таким образом алгоритм позволяет преодолевать неоднозначности, вызванные
симметричностью объекта.
\Comment{Пояснить про полосу около границы и внутренность. Привести в
пристойный вид}
