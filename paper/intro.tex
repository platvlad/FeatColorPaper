\section{Введение}

Тема данной работы касается проблемы извлечения информации о перемещении
объекта в трехмерном пространстве по видео.

Отслеживание "--- \term{трекинг} "--- положения объекта в 3D требуется в
приложениях дополненной реальности\cite{Radkowski}, во многих задачах
робототехники\cite{Robotics} и других приложениях, где необходимо понимание
окружающего простанства по видео.
Одно из применений трекинг находит в создании видеоэффектов в кино, где часто
требуется отследить на видео некоторый объект с целью придания ему затем
дополнительных свойств\cite{Bugaev_2018_ECCV}.

Данная работа посвящена одной из задач, касающихся упомянутой проблемы "---
рекурсивному трекингу 6D-позиции заданного 3D-моделью объекта на RGB-видео.
Выражение «6D-позиция» происходит от того, что положение объекта в трехмерном
пространстве имеет шесть степеней свободы.
Входными данными данной задачи являются видео, снятое с единственной
камеры с известными параметрами, 3D-модель отслеживаемого объекта
и его позиция в трехмерном пространстве на начальном кадре видео.
Целью задачи является последовательное вычисление позиции объекта относительно
камеры в каждом кадре видео.

Наличие лишь одного кадра с доподлинно известной позицией объекта и
последовательное вычисление позиций являются важными отличительными
особенностями рассматриваемой задачи.
В то же время нередко рассматривается ее постановка, подразумевающая наличие
дополнительных \term{ключевых кадров} "--- нескольких изображений с известной
позицией объекта относительно камеры.
Дополнительная входная информация обычно дает возможность получать более
качественное решение, однако у требущих наличия ключевых кадров методов область
возможного применения несколько уже.
Тем не менее в основном методы решения задачи в обоих вариантах используют
общие идеи, поэтому при ссылках на связанные работы данное различие будет
упоминаться только в случае необходимости.

Рассматриваемая задача подразумевает, что отслеживаемый объект может быть
произвольным и на момент начала трекинга задано лишь одно (или небольшое
число, если есть ключевые кадры) изображение с известной позицией объекта,
что значительно усложняет возможность применения машинного обучения.
В связи с этим известные на текущий момент методы решения задачи опираются на
классические методы компьютерного зрения.
Их можно классифицировать в зависимости от того, на какого рода особенности
изображения они опираются "--- цветовые, точечные, граничные и т.\,п.
Различные особенности позволяют в разной степени противостоять возможным
случаям, представляющим существенные сложности для алгоритмов трекинга.
Такими являются захламленный фон, смазывание видео при быстром движении,
быстрая перемена освещения, предметы, перекрывающие объект,
однородная текстура объекта, его симметричная форма.

Некоторые алгоритмы\cite{Vacchetti2004,Lourakis2013,Pauwels2013}
используют точечные особенности изображения (ключевые точки).
Они выделяют на поверхности объекта небольшие хорошо различимые
участки\cite{AKAZE,SIFT,ShiAndTomasi} и отслеживают их перемещение от кадра
к кадру\cite{TomasiAndKanade,SIFT,PyrLK}.
Позиция объекта восстанавливается на основе знания о расположении данных
участков на кадре и соответствующих им точках на 3D-модели\cite{EPnP}.
Такие методы лучше других справляются с перекрытиями и симметричной формой,
но при этом требуют наличия неоднородной текстуры объекта.

В то же время многие объекты имеют характерную форму, которая позволяет
определять их позицию путем сопоставления контуров 3D-модели
и границ\cite{EdgesSurvey,CANNY} изображения объекта на кадре.
Такие методы\cite{RAPID,Marchand2003,Choi2012,Marchand2006,Klein2006,SeoHinterstoisser2014,WangZhong2015,Damen2012,VacchettiEdges2004}
позволяют производить трекинг даже если поверхность объекта окрашена однородно.
В то же время подобные алгоритмы нередко дают сбои при наличии захламленного
фона и при возникновении неоднозначностей в определении позиций
симметричных объектов.

В последние годы активно развиваются алгоритмы, основанные на
распределении цветов в отдельных областях
изображения\cite{PWP3D,Tjaden2017,Tjaden2018}.
Они собирают статистику того, насколько часто пиксели различных цветов
встречаются на проекции объекта и на окружающем фоне.
На каждом новом кадре позиция ищется таким образом, чтобы проекция объекта
наилучшим образом соответствовала собранной статистике.
Такие методы хорошо справляются с однородно окрашенными объектами, если их цвет
отличается от цвета фона, а также лучше других противостоят смазыванию при
быстром движении объектов.
В то же время они плохо справляются с симметричными объектами и неустойчивы к
резкой смене освещения.

Таким образом, различные типы методов по-разному справляются со сложными
условиями трекинга.
Там, где один сбивается, другой может работать стабильно.  В связи с этим
многие работы\cite{RegionPhotometric,ColorFeature2018,Bugaev_2018_ECCV}
посвящены комбинированным алгоритмам, совмещающим в себе различные типы
методов.

\Comment{Описать вклад работы}

В данной работе проведена попытка одновеменно исользовать инфомацию о
распределении цветов на проекции и фоне на предыдущих изображениях и информацию
о ключевых точках.
Подход на основе распределения цветов выбран, так как подходит для любых типов
текстур, устойчив при несильном смазывании и показывающий хорошие
результаты~\cite{Tjaden2018}.
Добавление к нему информации о ключевых точках позволяет разрешить
неопределённость, связанную с симметричностью объектов, и кроме того привлекает
информацию о внутренней области проекции, тогда как алгоритмы, основанные на
цветах, используют в основном информацию о пикселях вокруг контура.


%При этом, в случае, если изображение зашумлено, то значения дескрипторов, соответствующие одним и тем же точкам, могут сильно изменяться, и это приведёт к зашумлённости функции ошибки  и наличию локальных минимумов.

Предлагаемый в данной статье подход отличается тем, что использует не все
пиксели из внутренней части проекции, а только ключевые точки, уверенность в
положении которых может быть выше, чем в постоянстве дескрипторов, описывающих
произвольные точки на текстуре.

Рассматриваются два способа совмещения.
Один из них использует метод ключевых точек для первоначальной оценки положения
объекта и инициализации цветового метода, то есть применяет алгоритмы
последовательно.
Второй способ строит совместную функцкию энергии, основанную как на цветовых
гистограммах, так и на особых точках и минимизирует её.

\Comment{Описать структуру статьи}

Дальнейшие главы организованы следующим образом: во второй главе метод ключевых
точек и метод цветовых гистограмм описаны по отедльности, в третьей главе
раскрываются способы их объединения, в четвёртой описывается тестирование и
оценка результатов.
