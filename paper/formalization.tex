\section{Формальная постановка задачи и используемые обозначения}\label{formalization}

Алгоритм трекинга принимает на вход последовательность кадров "---
RGB-изображений
$
    \Img_i \colon \ImgDom \to \Real^3
$,
где $\ImgDom \subset \Real^2$ "--- область определения изображения.
Таким образом, $\Img_i(\vect{u})$ "--- значение цвета, которое принимает
изображение $\Img_i$ в точке $\vect{u} \in \ImgDom$.
В тех случаях, когда номер кадра неважен, он будет опускаться.

Внутренние параметры камеры, на которую сняты входные кадры, задаются матрицей
\begin{equation}\label{eqn:camintr}
    \CamIntr = \begin{bmatrix}
        f_x & 0   & c_x \\
        0   & f_y & c_y \\
        0   & 0   & 1
    \end{bmatrix}
    \text{.}
\end{equation}
Они одинаковы для всех кадров видео.

Форма отслеживаемого объекта задается 3D-моделью $\Mesh$.
Определим ее как пару $(\MeshV, \MeshF)$, где
$\MeshV \subset \Real^3$~--- конечное множество вершин модели, а
$\MeshF$~--- множество троек вершин, задающих грани модели.
При взгляде на лицевую сторону грани вершины будут располагаться против часовой
стрелки.

Позиция объекта в кадре $i$ имеет шесть степеней свободы и описывается матрицей
\begin{equation}\label{eqn:modelpose}
    \Pose_i = \left[ \begin{array}{ccc|c}
          & \RotMat_i &   & \TrVec_i \\
        \hline
        0 & 0         & 0 & 1
    \end{array} \right] \in \SE
    \,\text{,}
\end{equation}
где матрица $\RotMat_i \in \SO$ задает ориентацию модели, а вектор
$\TrVec_i \in \Real^3$~--- сдвиг.
На вход алгоритма принимается положение объекта на начальном кадре,
а результатом его работы являются последовательно вычисленные позиции
в остальных кадрах.

Проекция $\vect{u} \in \Real^2$ точки на поверхности модели
$\vect{x} \in \Real^3$ описывается стандартной моделью камеры
\begin{equation}\label{eqn:proj}
    \homv{u} = \CamIntr \cdot [\RotMat_i \mid \TrVec_i] \cdot \homv{x}
    \,\text{,}
\end{equation}
где $\homv{u}$ и $\homv{x}$~--- вектора $\vect{u}$ и $\vect{x}$ в однородных
координатах. Функцию, выполняющую проекцию $\vect{x} \mapsto \vect{u}$ в
позиции $\Pose_i$ будем обозначать как $\projAt{\Pose_i}$ или просто $\proj$,
если позиция понятна из контекста.

Область изображения, соответствующая проекции 3D-модели объекта
в текущей позиции, называется маской проекции и обозначается $\FgProj$.
Остальная часть изображения обозначается
$\BgProj = \ImgDom \setminus \FgProj$.
Границей между $\FgProj$ и $\BgProj$ является контур $\Contour$,
задающий
$\CDist \colon \ImgDom \to \Real$ "---
функцию расстояния до контура со знаком
\begin{equation*}
    \CDist(\vect{u}) =
    \begin{cases}
        d(\Contour, \vect{u})  & \vect{u} \in \FgProj \\
        -d(\Contour, \vect{u})  & \vect{u} \in \BgProj
    \end{cases}
    \,\text{,}
\end{equation*}
где $d(\Contour, \vect{u})$ "--- кратчайшее расстояние от точки $\vect{u}$
до контура $\Contour$.

При описании цветового трекинга будет рассматриваться окрестность контура.
Эта подобласть $\Omega$ обозначается как 

\begin{equation*}
	\CtLocal = \{ \vect{u} \in \Omega \mid |\CDist(\vect{u})| < r \}
\end{equation*}

Её часть, входящая в маску проекции, обозначается как $\CtLocalFg = \CtLocal \cap \FgProj$.
Часть, принадлежащая фону, обозначается $\CtLocalBg = \CtLocal \setminus \CtLocalFg$

Области кадра, соответствующие истинному изображению объекта и истинному фону,
обозначаются как $\FgTrue$ и $\BgTrue$ соответственно.
