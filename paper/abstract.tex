\begin{abstract}

В данной работе рассматривается комбинированный подход к отслеживанию позиции
заданного 3D-моделью объекта в трехмерном пространстве по RGB-видео.
Отслеживание выполняется последовательно: положение объекта на начальном кадре
видео известно, затем оно вычисляется на втором, потом на третьем и так далее.
Предлагается способ совмещения двух методов трекинга, использующих различные
особенности изображения.
На первом этапе применяется алгоритм вычисления позиции с помощью ключевых
точек на поверхности объекта.
Наряду с этим производится попытка предсказать позицию путем экстраполяции по
двум предыдущим кадрам.
Затем из этих двух позиций выбирается лучшая с точки зрения цветовых
характеристик объекта и фона.
Она используется в качестве отправной точки для оптимизации позиции методом,
опирающимся на данные цветовые характеристики.
Затем производится согласование состояния точечного алгоритма с полученной
позицией.
Предлагаемые модификации цветового алгоритма позволяют сократить потребление
оперативной памяти.
Предлагаемый комбинированный метод позволяет лучше преодолевать сложные
ситуации, чем составляющие его алгоритмы по отдельности.
Эффективность подхода демонстрируется путем экспериментов на открытом
тестовом наборе данных.

\end{abstract}
