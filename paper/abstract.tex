\begin{abstract}
\Comment{TODO: переписать после того, как будет готово все остальное.}

Трекинг трёхмерных объектов по видеопотоку -- важная задача компьютерного
зрения.
Применяемые для её решения алгоритмы могут использовать различные виды
информации на изображениях.
Наиболее распространены подходы, которые выделяют границы на изображении,
используют ключевые точки, значения цвета и другую информацию.
Они обладают своими преимуществами и недостатками, проявляющимися в различных
условиях.
Целью данной статьи является комбинирование подхода, использующего ключевые
точки с подходом, основанным на распределении цветов.
Рассмотрены два различных алгоритма совмещения методов.
В одном из них метод, основанный на ключевых точках, используется для получения
первого приближения позиции объекта, которая дальше уточняется методом,
использующим цветовые гистограммы.
В другом -- вводится совместная функция энергии, учитывающая как информацию о
ключевых точках, так и информацию о цвете.
Комбинирование двух разных классов методов позволяет использовать преимущества
обоих подходов и повысить устойчивость трекинга в неблагоприятных условиях.
Общая точность работы на некоторых примерах также превосходит точность трекинга
совмещаемыми методами по отдельности, что продемонстрировано при тестировании
на датасете OPT.
\end{abstract}
